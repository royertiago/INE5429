\documentclass{article}
\usepackage[utf8]{inputenc}
\usepackage[brazil]{babel}
\usepackage{amsmath}
\usepackage{amssymb}

\title{
    INE5429 --- Segurança em Computação \\
    Relatório I --- Geração de Números Primos
}
\author{
    Tiago Royer - 12100776
}

\date{16 de setembro de 2015}

\begin{document}

\maketitle

O objetivo deste trabalho é gerar números primos grandes.
Números primos possuem sua importância em protocolos de segurança
porque o problema de encontrar os fatores primos de um número não-primo
é computacionalmente intratável para números grandes,
cujos fatores primos possuem milhares de bits.
Portanto, ao gerar dois números primos grandes $p$ e $q$,
mesmo que disponibilizemos para o mundo todo o produto $pq$
podemos confiar que nenhum atacante consegue fatorar este número em tempo hábil.

O trabalho pedia que implementássemos um gerador de números aleatórios
e um testador probabilístico de primalidade,
e os usasse para computar uma tabela de números primos com tamanhos arbitrários.

\end{document}
