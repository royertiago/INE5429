\documentclass{article}
\usepackage[utf8]{inputenc}
\usepackage[brazil]{babel}
\usepackage{amsmath}

\title{
    INE5429 --- Segurança em Computação \\
    Relatório III --- Protocolo Diffie-Hellman
}
\author{
    Tiago Royer - 12100776
}

\date{05 de outubro de 2015}

\begin{document}
\maketitle

O objetivo deste trabalho era estudar e implementar
o protocolo de troca de chaves Diffie-Hellman.

Todo o código está disponível em \verb"https://github.com/royertiago/INE5429".

\section{Descrição do protocolo}

Neste protocolo,
o objetivo é construir um número $k$ que será comum às duas partes envolvidas
(que chamaremos de Alice e Bob),
mas que seja desconhecido para quem quer que esteja ouvindo o canal de comunicação.

Sejam $p$ um número primo e $a$ uma raiz primitiva módulo $p$;
estes dois números serão usados para definir o corpo sobre o qual
as operações do protocolo serão feitas,
e devem ser conhecidos tando por Alice quanto por Bob de antemão.

Alice gera um número $A$ e Bob gera um número $B$.
Os números $A$ e $B$ são mantidos privados por Alice e Bob,
respectivamente.
Então, Alice e Bob enviarão,
respectivamente, os números
\begin{equation*}
    a^A \bmod p \quad \text{e} \quad a^B \bmod p
\end{equation*}
pelo canal inseguro.

Usando exponenciação modular,
é rápido obter os valores $a^A \bmod p$ e $a^B \bmod p$
a partir de $A$ e $B$, respectivemente;
entretanto,
mesmo que $a$, $p$, $a^A \bmod p$ e $a^B \bmod p$ sejam descobertos,
ainda assim é computacionalmente intratável
obter os valores $A$ e $B$ individualmente.
(Este problema corresponde a uma instância do log discreto,
o qual não possui algoritmo rápido conhecido.)

Finalmente, Alice calcula $(a^B)^A \equiv a^{AB} \pmod p$
usando seu número $A$,
e Bob calcula $(a^A)^B \equiv a^{AB} \pmod p$,
usando seu número $B$.
Agora,
Alice e Bob compartilham o número $k = a^{AB}$.
Observe que, ainda assim,
não é possível calcular rapidamente o valor $k$ sem ter acesso a $A$ ou a $B$;
de fato, Alice sequer sabe o valor $B$
e Bob não conhece o valor $A$.
Entretanto,
ambos conhecem o número $k$,
e apenas eles conhecem este número,
que pode ser usado futuramente como uma chave
para um algoritmo de criptografia simétrico, por exemplo.

\section{Exemplo numérico}

O primo utilizado será $p = 2017$
e a raiz primitiva módulo $p$ escolhida é $a = 5$.

Usando algum gerador de números aleatórios,
Alice gera o número $A = 859$,
e Bob gera o número $B = 34$.
Os valores correspondentes são
\begin{align*}
    a^A = 5^{859} &\equiv 315 \pmod p \\
    a^B = 5^{34} &\equiv 1250 \pmod p
\end{align*}

Os valores $315$ e $1250$ são enviados pela rede.
Alice, então,
eleva o valor $1250$ ao número $A$, obtendo
\begin{equation*}
    1250^A = 1250^{859} \equiv 696 \pmod p.
\end{equation*}
Bob, por sua vez,
eleva o valor $315$ ao número $34$, obtendo
\begin{equation*}
    315^B = 315^{34} \equiv 696 \pmod p.
\end{equation*}

Alice e Bob agora compartilham do valor $696$.

\section{Implementação}

O protocolo foi encapsulado numa classe,
que cuida do processo de gerar as chaves secretas e a chave comum.
Este código está em \verb"protocols/diffie_hellman.hpp".

\begin{verbatim}
template< typename T = mpz_class >
class diffie_hellman {
    T prime, primitive_root;
    T private_number;
    T public_number;
    T partner_public_number;
    T common_secret;

public:
    diffie_hellman( T prime, T primitive_root ):
        prime( prime ),
        primitive_root( primitive_root )
    {}

    /* Generate our private number,
     * that will be used to construct the common secret number.
     */
    template< typename RNG >
    void generate_private_number( RNG & rng ) {
        // TODO: Make this GMP-independent
        int bytes = (mpz_sizeinbase( prime.get_mpz_t(), 2 ) + 7)/8;
        private_number = rng::gmp_generate( rng, bytes );
        private_number %= prime;

        public_number = math::pow_mod( primitive_root, private_number, prime );
    }

    /* Returns the public number that must be sent to our partner.
     */
    T get_public_number() const {
        return public_number;
    }

    /* Informs the public number of our partner.
     */
    void set_partner_public_number( T t ) {
        partner_public_number = t;
        common_secret = math::pow_mod( t, private_number, prime );
    }

    /* Returns the secret number which we share with our partner.
     */
    T get_common_secret() const {
        return common_secret;
    }
};
\end{verbatim}

A função \verb"rng::gmp_generate" adapta o gerador de números aleatórios
para gerar um número da classe \verb"mpz_class",
a classe dos inteiros da biblioteca \verb"GMP".

O programa \verb"diffie_hellman.cpp" provê uma interface
para acessar esta classe.
Este programa assume que $p$ e $a$ já estão calculados.

\subsection{Exemplo de execução}

Usando o gerador de primos do primeiro trabalho,
gerei o número
\begin{align*}
    p = & 988602409265782276575624507120340176672860211163704 \\
        & 64121339065271298938017858524727018293745742964289,
\end{align*}
de $101$ dígitos.
$p-1$ pode ser fatorado como
\begin{align*}
    p-1 = & 2^6 * 3 * 61 * 157 * 3001 * 5827 * 60869 * 1264559 * 5508074323 * \\
          & 7251784326360321507227567834331220107621459114376891631201987573377
\end{align*}

O gerador de raízes primitivas do trabalho 2 determinou que
$a = 11$ é uma raiz primitiva módulo $p$.

Usando duas instâncias deste programa,
podemos simular a troca de chaves.
As primeiras duas linhas contém o comando digitado no terminal.
As duas linhas seguintes contém a chave pública gerada por Alice.
Na linha 5,
eu digitei o valor gerado por Bob,
e as duas últimas linhas contém a chave em comum $k$.

\begin{verbatim}
$ ./diffie_hellman 98860240926578227657562450712034017667286021116
370464121339065271298938017858524727018293745742964289 11
Our public number: 19868465663106381873323139927395425510614517607
737460628117410148368935680408066910484493133083215468
Input parter's public number: 299116422830129556719122640752664695
49749835521112148230672740127509445841723393926173765489992703076
Common secret is 9928874927765485952788708835063095327386488740880
742076014608966023101213546194971044410442751128001
\end{verbatim}

As linhas abaixo contém a execução da segunda instância,
representando Bob.
(Foi daqui que eu tirei a chave pública digitada acima.)

\begin{verbatim}
$ ./diffie_hellman 98860240926578227657562450712034017667286021116
370464121339065271298938017858524727018293745742964289 11
Our public number: 29911642283012955671912264075266469549749835521
112148230672740127509445841723393926173765489992703076
Input parter's public number: 198684656631063818733231399273954255
10614517607737460628117410148368935680408066910484493133083215468
Common secret is 9928874927765485952788708835063095327386488740880
742076014608966023101213546194971044410442751128001
\end{verbatim}

\end{document}
