\documentclass{article}
\usepackage[utf8]{inputenc}
\usepackage[brazil]{babel}
\usepackage{amsmath}

\title{
    INE5429 --- Segurança em Computação \\
    Relatório III --- Protocolo Diffie-Hellman
}
\author{
    Tiago Royer - 12100776
}

\date{05 de outubro de 2015}

\begin{document}
\maketitle

O objetivo deste trabalho era estudar e implementar
o protocolo de troca de chaves Diffie-Hellman.

Todo o código está disponível em \verb"https://github.com/royertiago/INE5429".

\section{Descrição do protocolo}

Neste protocolo,
o objetivo é construir um número $k$ que será comum às duas partes envolvidas
(que chamaremos de Alice e Bob),
mas que seja desconhecido para quem quer que esteja ouvindo o canal de comunicação.

Sejam $p$ um número primo e $a$ uma raiz primitiva módulo $p$;
estes dois números serão usados para definir o corpo sobre o qual
as operações do protocolo serão feitas,
e devem ser conhecidos tando por Alice quanto por Bob de antemão.

Alice gera um número $A$ e Bob gera um número $B$.
Os números $A$ e $B$ são mantidos privados por Alice e Bob,
respectivamente.
Então, Alice e Bob enviarão,
respectivamente, os números
\begin{equation*}
    a^A \bmod p \quad \text{e} \quad a^B \bmod p
\end{equation*}
pelo canal inseguro.

Usando exponenciação modular,
é rápido obter os valores $a^A \bmod p$ e $a^B \bmod p$
a partir de $A$ e $B$, respectivemente;
entretanto,
mesmo que $a$, $p$, $a^A \bmod p$ e $a^B \bmod p$ sejam descobertos,
ainda assim é computacionalmente intratável
obter os valores $A$ e $B$ individualmente.
(Este problema corresponde a uma instância do log discreto,
o qual não possui algoritmo rápido conhecido.)

Finalmente, Alice calcula $(a^B)^A \equiv a^{AB} \pmod p$
usando seu número $A$,
e Bob calcula $(a^A)^B \equiv a^{AB} \pmod p$,
usando seu número $B$.
Agora,
Alice e Bob compartilham o número $k = a^{AB}$.
Observe que, ainda assim,
não é possível calcular rapidamente o valor $k$ sem ter acesso a $A$ ou a $B$;
de fato, Alice sequer sabe o valor $B$
e Bob não conhece o valor $A$.
Entretanto,
ambos conhecem o número $k$,
e apenas eles conhecem este número,
que pode ser usado futuramente como uma chave
para um algoritmo de criptografia simétrico, por exemplo.

\end{document}
