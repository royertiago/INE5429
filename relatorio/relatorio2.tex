\documentclass{article}
\usepackage[utf8]{inputenc}
\usepackage[brazil]{babel}
\usepackage{amsmath}
\usepackage{amssymb}
\usepackage{amsthm}
\newtheorem{theorem}{Teorema}
\theoremstyle{definition}
\newtheorem{definition}{Definição}
\newtheorem{example}{Exemplo}

\title{
    INE5429 --- Segurança em Computação \\
    Relatório II --- Raízes Primitivas
}
\author{
    Tiago Royer - 12100776
}

\date{28 de setembro de 2015}

\begin{document}

\maketitle

\section{Definição}

\begin{definition}
    Dado um primo $p$,
    um número $a$ é uma \emph{raíz primitiva módulo $p$}
    se
    \begin{equation*}
        \{a^1 \mod p, a^2 \mod p, \dots, a^n \mod p, \dots \} = \{1, 2, \dots, p-1\}.
    \end{equation*}
    Isto é, se as potências $a^1$, $a^2$, $a^3$ e assim por diante
    gerarem todos os números entre $1$ e $p-1$.
\end{definition}

\begin{example}
    Seja $p = 11$.
    O número $10$ não é uma raíz primitiva módulo $11$ pois
    \begin{align*}
        10^1 &\equiv 10 \mod 11 \\
        10^2 &\equiv 1 \mod 11 \\
        10^3 &\equiv 10 \mod 11 \\
        ...
    \end{align*}
    Após o expoente sendo testado atingir $1$,
    os números irão se repetir,
    portanto podemos ``abortar'' o teste.

    O número $5$ também não é:
    \begin{align*}
        5^1 &\equiv 5 \mod 11 \\
        5^2 &\equiv 3 \mod 11 \\
        5^3 &\equiv 4 \mod 11 \\
        5^4 &\equiv 9 \mod 11 \\
        5^4 &\equiv 1 \mod 11
    \end{align*}

    Já o número $2$ é uma raíz primitiva módulo $11$:
    \begin{align*}
        2^1 &\equiv 2 \mod 11 \\
        2^2 &\equiv 4 \mod 11 \\
        2^3 &\equiv 8 \mod 11 \\
        2^4 &\equiv 5 \mod 11 \\
        2^5 &\equiv 10 \mod 11 \\
        2^6 &\equiv 9 \mod 11 \\
        2^7 &\equiv 7 \mod 11 \\
        2^8 &\equiv 3 \mod 11 \\
        2^9 &\equiv 6 \mod 11 \\
        2^{10} &\equiv 1 \mod 11
    \end{align*}
\end{example}

\end{document}
