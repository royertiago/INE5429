\documentclass{article}
\usepackage[utf8]{inputenc}
\usepackage[brazil]{babel}
\usepackage{amsmath}
\usepackage{amssymb}
\usepackage{amsthm}
\newtheorem{theorem}{Teorema}
\theoremstyle{definition}
\newtheorem{definition}{Definição}
\newtheorem{example}{Exemplo}

\title{
    INE5429 --- Segurança em Computação \\
    Relatório II --- Raízes Primitivas
}
\author{
    Tiago Royer - 12100776
}

\date{28 de setembro de 2015}

\begin{document}

\maketitle

\section{Definição}

\begin{definition}
    Dado um primo $p$,
    um número $a$ é uma \emph{raiz primitiva módulo $p$}
    se
    \begin{equation*}
        \{a^1 \mod p, a^2 \mod p, \dots, a^n \mod p, \dots \} = \{1, 2, \dots, p-1\}.
    \end{equation*}
    Isto é, se as potências $a^1$, $a^2$, $a^3$ e assim por diante
    gerarem todos os números entre $1$ e $p-1$.
\end{definition}

\begin{example}
    Seja $p = 11$.
    O número $10$ não é uma raiz primitiva módulo $11$ pois
    \begin{align*}
        10^1 &\equiv 10 \mod 11 \\
        10^2 &\equiv 1 \mod 11 \\
        10^3 &\equiv 10 \mod 11 \\
        ...
    \end{align*}
    Após o expoente sendo testado atingir $1$,
    os números irão se repetir,
    portanto podemos ``abortar'' o teste.

    O número $5$ também não é:
    \begin{align*}
        5^1 &\equiv 5 \mod 11 \\
        5^2 &\equiv 3 \mod 11 \\
        5^3 &\equiv 4 \mod 11 \\
        5^4 &\equiv 9 \mod 11 \\
        5^4 &\equiv 1 \mod 11
    \end{align*}

    Já o número $2$ é uma raiz primitiva módulo $11$:
    \begin{align*}
        2^1 &\equiv 2 \mod 11 \\
        2^2 &\equiv 4 \mod 11 \\
        2^3 &\equiv 8 \mod 11 \\
        2^4 &\equiv 5 \mod 11 \\
        2^5 &\equiv 10 \mod 11 \\
        2^6 &\equiv 9 \mod 11 \\
        2^7 &\equiv 7 \mod 11 \\
        2^8 &\equiv 3 \mod 11 \\
        2^9 &\equiv 6 \mod 11 \\
        2^{10} &\equiv 1 \mod 11
    \end{align*}
\end{example}

\section{Geração de raízes primitiva módulo $p$}

Conforme o exemplo anterior,
assim que a sequência
\begin{equation*}
    a^1, a^2, a^3 \ldots
\end{equation*}
atingir o valor $1$, módulo $p$,
a sequência começa a se repetir.
O primeiro $n$ para o qual $a^n \equiv 1 \mod p$
é chamado de \emph{ordem de $a$ módulo $p$}.
Temos o seguinte teorema.

\begin{theorem}
    A ordem de qualquer número $a$ módulo $p$ divide $\phi(p)$.
\end{theorem}

$\phi$ é a função totiente de Euler;
se $p$ é primo, $\phi(p) = p-1$.

Este teorema nos dá o seguinte algoritmo
(extraído de \cite{PrimitiveRootsStackOverflow})
para determinar se $a$ é uma raiz primitiva módulo $p$.

\begin{enumerate}
    \item Fatore $p-1$ nos primos $p_1, p_2, \ldots, p_k$.
    \item Teste, para cada $i \leq k$, se
        \begin{equation*}
            a^{(p-1)/p_i} \equiv 1 \mod p
        \end{equation*}
        Se sim, pare: $a$ não é uma raiz primitiva módulo $p$.
    \item Se $a$ passar em todos os testes,
        $a$ é uma raiz primitiva módulo $p$.
\end{enumerate}

Observe que não precisamos calcular $a^d \mod p$
para todos os divisores de $\phi(p)$;
todos os divisores de $\phi(p)$ terão os mesmos fatores primos de $\phi(p)$,
portanto caso $a^d \equiv 1 \mod p$ para algum divisor $d$,
este $d$ divide algum dos números na forma $(p-1)/p_i$ testados no algoritmo,
portanto $a$ será ``pego'' ao testar $(p-1)/p_i$.

Para encontrar uma raiz primitiva módulo $p$,
então,
basta testar todos os números entre $2$ e $p-1$.

\begin{example}
    Vamos encontrar uma raiz primitiva módulo $41$.
    $\phi(41) = 40 = 2^3*5$,
    portanto precisamos testar as potências $40/2 = 20$ e $40/5 = 8$.

    Testando com $a = 2$, temos $a^{20} \equiv 1 \mod 41$,
    portanto $2$ não é uma raiz primitiva módulo $41$.

    Testando com $a = 3$, temos $a^{30} \equiv 40 \mod 41$,
    mas $a^{16} \equiv 1 \mod 41$,
    portanto $3$ também não é uma raiz primitiva módulo $41$.

    Podemos ignorar o caso $a = 4 = 2^2$,
    pois, se $4$ fosse uma raiz primitiva módulo $41$,
    $2$ também o seria.
    (Podemos, de fato, ignorar todas as potências perfeitas;
    entretanto,
    como potências perfeitas são raras,
    não há por que testar explicitamente contra elas na execução do algoritmo.)

    Testando com $a = 5$, temos $5^{20} \equiv 1 \mod 41$,
    portanto $5$ também não é raiz primitiva módulo $41$.

    Finalmente, com $a = 6$, temos $6^{20} \equiv 40 \mod 41$
    e $6^8 \equiv 10 \mod 41$,
    portanto $6$ é uma raiz primitiva módulo $41$.
\end{example}

\bibliographystyle{plain}
\bibliography{bibliografia}

\end{document}
