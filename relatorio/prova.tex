\documentclass{article}

\usepackage[utf8]{inputenc}
\usepackage[T1]{fontenc}
\usepackage{amsmath}
\usepackage{booktabs}

\begin{document}

\title{Resolução da prova}
\author{Tiago Royer}
\date{4 de novembro de 2015}
\maketitle

Meu número de matrícula é $12100776$,
portanto, de acordo com a tabela,
$p = 19$ e $v = 101$.

\textbf{
    Questão 1 ---
    Considere que você deseja cifrar e decifrar mensagens de até 9 bits,
    de forma o mais eficiente possível,
    usando o algoritmo RSA.
}

\textbf{
    (a) Nesse cenário, gere um par de chaves RSA,
    onde um dos primos deve ser $p$
    e o expoente público deve ser $13$.
}

Para cifrar números de até $9$ bits,
é necessário que o módulo $n$ do algoritmo RSA
seja maior ou igual a $2^9-1 = 511$.
O primeiro primo $q$ tal que
\begin{equation*}
    pq \geq 511
\end{equation*}
é $q = 29$,
portanto escolheremos este número para ser $q$.

O expoente público é $13$;
precisamos achar seu inverso módulo $\phi(pq) = (19-1)*(29-1) = 504$.
Aplicando o Euclides extendido, temos
\begin{align*}
    10 &= 1*504 - 38*13 \\
    3 &= 1*13 - 1*10 \\
    1 &= 1*10 - 3*3; \\
    1 &= 1*10 - 3*(1*13 - 1*10) \\
    1 &= 4*10 - 3*13 \\
    1 &= 4*(1*504 - 38*13) - 3*13 \\
    1 &= 4*504 - 155*13
\end{align*}

Portanto,
\begin{align*}
    (-155) * 13 &= 1 \pmod {504} \\
    349 * 13 &= 1 \pmod {504}
\end{align*}

Assim,
calculamos que $349$ é a inversa modular de $13$;
portanto, a chave privada correspondente à chave pública $(13, 551)$ é
\begin{equation*}
    (349, 551).
\end{equation*}

\textbf{
    (b) Usando a chave privada, cifre a mensagem $M = 5$.
}

$349 = 2^8 + 2^6 + 2^4 + 2^3 + 2^2 + 2^0$.
Calculando os expoentes $5^{2^i} \bmod 551$ para $0 \leq i \leq 8$,
temos

\begin{tabular}{r r r r}
    $i$ & $2^i$ & $(5^{2^{i-1}})^2$ & $5^{2^i} \bmod 551$ \\
    \toprule
    0 & 1 & - & 5 \\
    1 & 2 & $5 \times 5 = 25$ & 25 \\
    2 & 4 & $25 \times 25 = 625$ & 74 \\
    3 & 8 & $74 \times 74 = 5476$ & 517 \\
    4 & 16 & $517 \times 517 = 267289$ & 54 \\
    5 & 32 & $54 \times 54 = 2916$ & 161 \\
    6 & 64 & $161 \times 161 = 25921$ & 24 \\
    7 & 128 & $24 \times 24 = 576$ & 25 \\
    8 & 256 & $25 \times 25 = 625$ & 74 \\
\end{tabular}
\\[1em]
Agora,
\begin{align*}
    5^{349} &= 5^{256} * 5^{64} * 5^{16} * 5^8 * 5^4 * 5 \pmod{551} \\
            &= 74 * 24 * 54 * 517 * 74 * 5 \pmod{551} \\
            &= 35 \pmod{551}.
\end{align*}

\end{document}
