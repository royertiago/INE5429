\documentclass{article}

\usepackage[utf8]{inputenc}
\usepackage[T1]{fontenc}
\usepackage{amsmath}
\usepackage{booktabs}

\begin{document}

\title{Resolução da prova}
\author{Tiago Royer}
\date{4 de novembro de 2015}
\maketitle

Meu número de matrícula é $12100776$,
portanto, de acordo com a tabela,
$p = 19$ e $v = 101$.

\textbf{
    Questão 1 ---
    Considere que você deseja cifrar e decifrar mensagens de até 9 bits,
    de forma o mais eficiente possível,
    usando o algoritmo RSA.
}

\textbf{
    (a) Nesse cenário, gere um par de chaves RSA,
    onde um dos primos deve ser $p$
    e o expoente público deve ser $13$.
}

Para cifrar números de até $9$ bits,
é necessário que o módulo $n$ do algoritmo RSA
seja maior ou igual a $2^9-1 = 511$.
O primeiro primo $q$ tal que
\begin{equation*}
    pq \geq 511
\end{equation*}
é $q = 29$,
portanto escolheremos este número para ser $q$.

O expoente público é $13$;
precisamos achar seu inverso módulo $\phi(pq) = (19-1)*(29-1) = 504$.
Aplicando o Euclides extendido, temos
\begin{align*}
    10 &= 1*504 - 38*13 \\
    3 &= 1*13 - 1*10 \\
    1 &= 1*10 - 3*3; \\
    1 &= 1*10 - 3*(1*13 - 1*10) \\
    1 &= 4*10 - 3*13 \\
    1 &= 4*(1*504 - 38*13) - 3*13 \\
    1 &= 4*504 - 155*13
\end{align*}

Portanto,
\begin{align*}
    (-155) * 13 &= 1 \pmod {504} \\
    349 * 13 &= 1 \pmod {504}
\end{align*}

Assim,
calculamos que $349$ é a inversa modular de $13$;
portanto, a chave privada correspondente à chave pública $(13, 551)$ é
\begin{equation*}
    (349, 551).
\end{equation*}

\textbf{
    (b) Usando a chave privada, cifre a mensagem $M = 5$.
}

$349 = 2^8 + 2^6 + 2^4 + 2^3 + 2^2 + 2^0$.
Calculando os expoentes $5^{2^i} \bmod 551$ para $0 \leq i \leq 8$,
temos

\begin{tabular}{r r r r}
    $i$ & $2^i$ & $(5^{2^{i-1}})^2$ & $5^{2^i} \bmod 551$ \\
    \toprule
    0 & 1 & - & 5 \\
    1 & 2 & $5 \times 5 = 25$ & 25 \\
    2 & 4 & $25 \times 25 = 625$ & 74 \\
    3 & 8 & $74 \times 74 = 5476$ & 517 \\
    4 & 16 & $517 \times 517 = 267289$ & 54 \\
    5 & 32 & $54 \times 54 = 2916$ & 161 \\
    6 & 64 & $161 \times 161 = 25921$ & 24 \\
    7 & 128 & $24 \times 24 = 576$ & 25 \\
    8 & 256 & $25 \times 25 = 625$ & 74 \\
\end{tabular}
\\[1em]
Agora,
\begin{align*}
    5^{349} &= 5^{256} * 5^{64} * 5^{16} * 5^8 * 5^4 * 5 \pmod{551} \\
            &= 74 * 24 * 54 * 517 * 74 * 5 \pmod{551} \\
            &= 35 \pmod{551}.
\end{align*}

\textbf{
    Questão 2 ---
    Considere o algoritmo de acordo de chaves de Diffie-Helamnn (DH)
    e seja $v$ o número primo parâmetro global do algoritmo.
}

\textbf{
    (a)  Determine os outros parâmetros públicos necessários ao DH.
}

Precisamos achar uma raíz primitiva de $101$.
$\phi(101) = 100 = 2^2 * 5^2$;
todas as raízes primitivas $\alpha$ satisfarão
\begin{align*}
    \alpha^{50} &\neq 1 \pmod{101};
    \alpha^{20} &\neq 1 \pmod{101}
\end{align*}

Iremos testar todos os números entre $2$ e $100$ até achar uma raíz primitiva.
Comecemos com o candidato $\alpha = 2$.

\begin{tabular}{r r r r}
    $i$ & $2^i$ & $(2^{2^{i-1}})^2$ & $2^{2^i} \bmod 101$ \\
    \toprule
    0 & 1 & - & 2 \\
    1 & 2 & $2 \times 2 = 4$ & 4 \\
    2 & 4 & $4 \times 4 = 16$ & 16 \\
    3 & 8 & $16 \times 16 = 256$ & 54 \\
    4 & 16 & $54 \times 54 = 2916$ & 88 \\
    5 & 32 & $88 \times 88 = 7744$ & 68 \\
    6 & 64 & $68 \times 68 = 4624$ & 79 \\
\end{tabular}

Agora,
\begin{align*}
    2^{50} &= 2^{32} * 2^{16} * 2^2 \pmod{101} \\
           &= 68 * 88 * 4 \pmod{101} \\
           &= 100 \pmod{101} \\
    2^{20} &= 2^{16} * 2^4 \pmod{101} \\
           &= 88 * 16 \pmod{101} \\
           &= 95 \pmod{101}
\end{align*}

Portanto, o candidato $\alpha = 2$ é uma raíz primitiva módulo $101$.
Assim,
os parâmetros públicos do algoritmo são $v = 101$ e $\alpha = 2$.

\textbf{
    (b)  Sabendo que as chaves secretas de Alice e Beto são $X_A = 5$
    $X_B = 50$, respectivamente,
    proceda ao acordo de chaves.
}

Alice calculará
\begin{equation*}
    Y_A = 2^{X_A} = 2^5 = 2^4 * 2 = 16*2 = 32 \pmod{101}.
\end{equation*}
Beto calculará
\begin{equation*}
    Y_B = 2^{X_B} = 2^{50} = 100 \pmod{101}.
\end{equation*}
($2^{50}$ foi calculado durante o calculo das raízes primitivas.)

Após Alice e Beto trocarem $Y_A$ e $Y_B$, eles calculam
\begin{align*}
    K = Y_B^{X_A} = 100^5 = 100^2 * 100^2 * 100 = 100 \pmod{101}; \\
    K = Y_A^{X_B} = 32^{50} = (2^5)^{50} = (2^{50})^5 = 100 \pmod{101}.
\end{align*}

\end{document}
