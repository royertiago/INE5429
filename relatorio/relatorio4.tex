\documentclass{article}
\usepackage[utf8]{inputenc}
\usepackage[brazil]{babel}
\usepackage{amsmath}

\title{
    INE5429 --- Segurança em Computação \\
    Relatório IV --- RSA
}
\author{
    Tiago Royer - 12100776
}

\date{14 de outubro de 2015}

\begin{document}
\maketitle

O objetivo deste trabalho era estudar e implementar
o algoritmo de criptografia do RSA.

Todo o código está disponível em \verb"https://github.com/royertiago/INE5429".

\section{Descrição do algoritmo}

De acordo com o teorema de Euler,
para todo inteiro $m$ (que entenderemos como sendo a mensagem),
todo inteiro $n$ e todo par de inteiros $e$ e $d$
(as chaves)
tais que
\begin{equation*}
    ed \equiv 1 \pmod{\varphi(n)},
\end{equation*}
temos
\begin{equation*}
    (m^e)^d \equiv m \pmod n.
\end{equation*}

Em essência,
o algoritmo RSA ``industrializa'' esta observação.
Gere dois primos grandes $p$ e $q$.
O valor $n$ será o produto destes dois primos:
\begin{equation*}
    n = pq
\end{equation*}
Se $p$ e $q$ forem distintos,
\begin{equation*}
    \phi(pq) = (p-1) * (q-1),
\end{equation*}
portanto, o valor de $\phi(n)$ é fácil de ser calculado.
Agora,
gere um valor $e$ e,
usando o algoritmo de Euclides extendido,
calcule sua inversa modular $d$ com relação a $\phi(n)$.
Se esta inversa modular existir,
geramos nosso par de chaves criptográficas:
$(e, n)$ é a chave pública,
que será usada para cifragem,
e $(d, n)$ é a chave privada,
que será usada para decifragem.

Dada uma mensagem $m$, o texto cifrado é
\begin{equation*}
    m^e \mod n
\end{equation*}

Dada a mensagem cifrada $m^e$, o texto original é
\begin{equation*}
    (m^e)^d \equiv m \pmod n.
\end{equation*}

\subsection{Exemplo numérico}

Escolha $p = 137$ e $q = 193$.
Temos $n = 26441$.
Para o valor $e$, da chave pública,
podemos escolher qualquer valor coprimo a $p-1$ e $q-1$;
por exemplo, $e = 977$.
Usando o Euclides extendido,
obtemos a equação
\begin{equation*}
    977 * 21809 \equiv 1 \pmod{\phi(n)};
\end{equation*}
portanto, $d = 21809$.

Temos a chave pública $(977, 26441)$, e a chave privada $(21809, 26441)$.

Para cifrar a mensagem $m = 4789$, por exemplo,
calculamos
\begin{equation*}
    4789 ^ {977} \bmod n = 22167.
\end{equation*}
Portanto, $22167$ é a mensagem cifrada.

Para decifrar, calculemos este valor elevado à chave privada.
\begin{equation*}
    22167 ^ {21809} \bmod n = 4789.
\end{equation*}

\end{document}
