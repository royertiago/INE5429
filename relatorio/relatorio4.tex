\documentclass{article}
\usepackage[utf8]{inputenc}
\usepackage[brazil]{babel}
\usepackage{amsmath}
\usepackage{amssymb}

\title{
    INE5429 --- Segurança em Computação \\
    Relatório IV --- RSA
}
\author{
    Tiago Royer - 12100776
}

\date{14 de outubro de 2015}

\begin{document}
\maketitle

O objetivo deste trabalho era estudar e implementar
o algoritmo de criptografia do RSA.

Todo o código está disponível em \verb"https://github.com/royertiago/INE5429".

\section{Descrição do algoritmo}

De acordo com o teorema de Euler,
para todo inteiro $m$ (que entenderemos como sendo a mensagem),
todo inteiro $n$ e todo par de inteiros $e$ e $d$
(as chaves)
tais que
\begin{equation*}
    ed \equiv 1 \pmod{\varphi(n)},
\end{equation*}
temos
\begin{equation*}
    (m^e)^d \equiv m \pmod n.
\end{equation*}

Em essência,
o algoritmo RSA ``industrializa'' esta observação.
Gere dois primos grandes $p$ e $q$.
O valor $n$ será o produto destes dois primos:
\begin{equation*}
    n = pq
\end{equation*}
Se $p$ e $q$ forem distintos,
\begin{equation*}
    \phi(pq) = (p-1) * (q-1),
\end{equation*}
portanto, o valor de $\phi(n)$ é fácil de ser calculado.
Agora,
gere um valor $e$ e,
usando o algoritmo de Euclides extendido,
calcule sua inversa modular $d$ com relação a $\phi(n)$.
Se esta inversa modular existir,
geramos nosso par de chaves criptográficas:
$(e, n)$ é a chave pública,
que será usada para cifragem,
e $(d, n)$ é a chave privada,
que será usada para decifragem.

Dada uma mensagem $m$, o texto cifrado é
\begin{equation*}
    m^e \mod n
\end{equation*}

Dada a mensagem cifrada $m^e$, o texto original é
\begin{equation*}
    (m^e)^d \equiv m \pmod n.
\end{equation*}

\subsection{Exemplo numérico}

Escolha $p = 137$ e $q = 193$.
Temos $n = 26441$.
Para o valor $e$, da chave pública,
podemos escolher qualquer valor coprimo a $p-1$ e $q-1$;
por exemplo, $e = 977$.
Usando o Euclides extendido,
obtemos a equação
\begin{equation*}
    977 * 21809 \equiv 1 \pmod{\phi(n)};
\end{equation*}
portanto, $d = 21809$.

Temos a chave pública $(977, 26441)$, e a chave privada $(21809, 26441)$.

Para cifrar a mensagem $m = 4789$, por exemplo,
calculamos
\begin{equation*}
    4789 ^ {977} \bmod n = 22167.
\end{equation*}
Portanto, $22167$ é a mensagem cifrada.

Para decifrar, calculemos este valor elevado à chave privada.
\begin{equation*}
    22167 ^ {21809} \bmod n = 4789.
\end{equation*}

\section{Implementação}

A maquinaria interna do RSA (dados $p$, $q$ e $e$)
está no arquivo \verb"protocols/rsa.hpp".
(O valor $e$ é chamado de $b$ no código.)

\begin{verbatim}
/* This structure represents the public key of the algorithm.
 * This data may be shared with everyone.
 */
template< typename T >
struct public_key {
    T b, n;

    // Encrypts the given number.
    T encrypt( T ) const {
        return math::pow_mod( x, b, n );
    }
};

/* This structure represents the private key of the algorithm.
 * This data should be kept private.
 */
template< typename T >
struct private_key {
    T a, n;

    // Decrypts the given number.
    T decrypt( T ) const {
        return math::pow_mod( y, a, n );
    }
};

/* Constructs the public key for the algorithm.
 * p and q must be primes.
 * b may be any number coprime with both p-1 and q-1.
 */
template< typename T >
public_key<T> build_public_key( T p, T q, T b ) {
    return { b, p * q };
}

/* Constructs the corresponding private key.
 * p, q and b must be the same for the function build_public_key.
 */
template< typename T >
private_key<T> build_private_key( T p, T q, T b ) {
    return { math::modular_inverse( b, T( (p-1) * (q-1) ) ), p * q };
}
\end{verbatim}

A função \verb"math::pow_mod" faz exponenciação modular.
\verb"math::modular_inverse" encapsula o algoritmo de Euclides extendido
para o cálculo de inversas modulares.

A geração dos números $p$, $q$ e $e$ está no arquivo \verb"rsa.cpp".
(Está omitido o parsing da linha de comando.)

\begin{verbatim}
int main( int argc, char ** argv ) {
    // Parse command line

    if( command_line::gen_key ) {
        std::ofstream public_key_file( command_line::public_key_file );
        std::ofstream private_key_file( command_line::private_key_file );
        rsa::public_key<mpz_class> public_key;
        rsa::private_key<mpz_class> private_key;

        rng::xorshift rng;
        mpz_class p = math::generate_prime_number(rng, command_line::key_size/2, 30);
        mpz_class q = math::generate_prime_number(rng, (command_line::key_size+1)/2, 30);
        mpz_class b = math::generate_prime_number(rng, 2*command_line::key_size/3, 30);
        // This number b is guaranteed to be coprime with both p and q.
        public_key = rsa::build_public_key(p, q, b);
        private_key = rsa::build_private_key(p, q, b);

        public_key_file << public_key << '\n';
        private_key_file << private_key << '\n';
    }
    else if( command_line::encrypt ) {
        std::fstream file( command_line::public_key_file );
        rsa::public_key<mpz_class> public_key;
        file >> public_key;

        mpz_class number;
        while( std::cin >> number )
            std::cout << public_key.encrypt(number) << '\n';
    } else {
        std::fstream file( command_line::private_key_file );
        rsa::private_key<mpz_class> private_key;
        file >> private_key;

        mpz_class number;
        while( std::cin >> number )
            std::cout << private_key.decrypt(number) << '\n';
    }

    return 0;
}
\end{verbatim}

\verb"math::generate_prime_number" é o código do trabalho 1.
Este algoritmo gera dois primos $p$ e $q$,
de tamanhos $\lfloor n/2 \rfloor$ e $\lceil n/2 \rceil$,
respectivamente.
O valor $d$ (chamado de \verb"b" no código)
é um número primo cujo tamanho está entre $\max(p, q)$ e $pq$,
portanto $d$ já é gerado coprimo a $\phi(n)$.

\subsection{Exemplo de execução}

O programa implementado é capaz de gerar chaves de tamanho arbitrário;
O script a seguir contém o processo de gerar uma chave de 4096 bits,
cifrar a mensagem $m = 1238926361552897$
(um dos fatores do oitavo número de Fermat),
e depois decifrá-la.

(Parte dos números fica oculta no PDF pois as chaves possuem muitos dígitos.)

\begin{verbatim}
$ ./rsa --gen-key 4096 --public public --private private
$ cat public
573755195778130424476191242260228768244062597815203148758486561303625442692648486993125114506365653275270225856913437866079484785724462207355991707703317372139871956689989042573307150526967897424302844649755487648768839065988058505926817039472228156756730397882364985672589918135800207678357132858916003089911079062400144171871807695414787334138303437986612173801049162448765948710171790143578207447787014298452361736509613560171031071888031484463498751207690918834547432978144522937828422788591413344114398163593425182596938510412984124175839543836165854889462738228139494294155263740098501038005244487198615645760823273382081999033797209688791328498975055467045247357886588933840556093769728659316577061977030185994916932386996706147662507655732584585610357229302535337727991262838608825540378805244261916846554499577447 776512913698329995009024714579379027539592372181141258737560971023536145527807719829915106189585788691280005933368555070982914796159032302120823035183048757141245181812868063249943245380640262216846429137891970260546557275198125000331939067473247707333663502682093298657259323762203468044161989861447906665772430329932637549591524915684148695192275200132945171609687813194197618286411366760691606499534143520333530958585775745565260479816271277089780754378777691157145206342976637995951349399149666793966052068879337159106526324285041852236003187137284525049554341271070822607440261438615176249676438059212814794909273077722416462051180795138973403877167383999106263146802340718240363316192178469838938048622643229193999237897392851849742298925380130530007829682752741607215727580121725029653929486404300785582629030784889688203281731301032486497825604141866138207565409640540580262942974524712264438815733951796296966074146872527231480534741856928965675124868366292072656396464188479699303485234094629437684052916231053626042114551831591589826930941513379846884728618016181499469479045727728582145533824098597547353371194817992538947279352919929508780133738782792439592518433895105520572641815227755026530112189268288542227091310503
$ cat private
367856506314145567799234587405829184427256762119883703611901203733626560574428876573304388847566595203669683786540541530818832474669147043127188559772313117016918213066328563876946820920559358844979413597765360022738390500683900235011760235839240463335456918573734412324103408317564722203805933650044071386524182812452189686783873150721871682048748399725736226933374613812330378782825007551902543591914571097511023004706322485275216223450534374152526618881976762213268227308776704685144377035871466260372586686187916704504695178094295842692309192920010386935743454371197475142685747299509633420718539169731597821604970682642531128108301288479723650642406564925673632360209935291935311658360712362612936091299087347580782601481490596876124735326351008824020776353261855597565000189075696424208667083672235923719985318079261993461773533472427642752754299260216483828065185124164667010040353666550850878173800717867138202258336863790549244790191723442061988855743353217232641259197850511250095586544253829560881074707315594148739660225893348453189557438803367982309666582044870706242007232431966166051108592377998963047552052799246618116696244107150700097386721909219305698982708550484129331755376425969402388600308056810070241648972527 776512913698329995009024714579379027539592372181141258737560971023536145527807719829915106189585788691280005933368555070982914796159032302120823035183048757141245181812868063249943245380640262216846429137891970260546557275198125000331939067473247707333663502682093298657259323762203468044161989861447906665772430329932637549591524915684148695192275200132945171609687813194197618286411366760691606499534143520333530958585775745565260479816271277089780754378777691157145206342976637995951349399149666793966052068879337159106526324285041852236003187137284525049554341271070822607440261438615176249676438059212814794909273077722416462051180795138973403877167383999106263146802340718240363316192178469838938048622643229193999237897392851849742298925380130530007829682752741607215727580121725029653929486404300785582629030784889688203281731301032486497825604141866138207565409640540580262942974524712264438815733951796296966074146872527231480534741856928965675124868366292072656396464188479699303485234094629437684052916231053626042114551831591589826930941513379846884728618016181499469479045727728582145533824098597547353371194817992538947279352919929508780133738782792439592518433895105520572641815227755026530112189268288542227091310503
$ ./rsa --public public --encrypt <<< 1238926361552897
550236520293135023139610064234111042377125011004916428822055392023022999833818738529724140802542083118107682010936086143277771056505223560740885790366758551231506073197364550866296101745191739313439305000235160758815423636180284422461790496894498286189827232248764077266525118320192664008034739529739367197008971188018425999448358438990675146629983517631568559572466663454930953290119352951995379256993574326376358269265139209051352582596858481974857401522711122871536763804123081440805686960698243181715384676156028655408883544136377895282516692139914437209526741282996608650474392017484677644753282923970554444150403232654557870869240202045500885640972086528602083619171899063443113630959791702592995730075030042257029781145007577272774103111679705738066561490231466254700624239205925185066466208425071167722738956263200623134939754475151224725249516593808277230391025381471774202542127767995206859754739816695524278921505378622057982242099019167478860009685955339017907870480240227966168741823222948581633627205110362759868960848256947494515655645204143936448389935789785229007094687650893522427851445357941731001324464768421575378652957650362416867534165949508824867311763363043915935000757999248708241059666147247725240870392620
$ ./rsa --private private --decrypt <<< 550236520293135023139610064234111042377125011004916428822055392023022999833818738529724140802542083118107682010936086143277771056505223560740885790366758551231506073197364550866296101745191739313439305000235160758815423636180284422461790496894498286189827232248764077266525118320192664008034739529739367197008971188018425999448358438990675146629983517631568559572466663454930953290119352951995379256993574326376358269265139209051352582596858481974857401522711122871536763804123081440805686960698243181715384676156028655408883544136377895282516692139914437209526741282996608650474392017484677644753282923970554444150403232654557870869240202045500885640972086528602083619171899063443113630959791702592995730075030042257029781145007577272774103111679705738066561490231466254700624239205925185066466208425071167722738956263200623134939754475151224725249516593808277230391025381471774202542127767995206859754739816695524278921505378622057982242099019167478860009685955339017907870480240227966168741823222948581633627205110362759868960848256947494515655645204143936448389935789785229007094687650893522427851445357941731001324464768421575378652957650362416867534165949508824867311763363043915935000757999248708241059666147247725240870392620
1238926361552897
\end{verbatim}

\section{Questionário}

\subsection{O que é a função totiente de Euler?}

A função $\varphi(n)$ calcula a quantidade de números menores que $n$
que são relativamente primos a $n$;
isto é,
quantos números do conjunto $\{0, 1, \dots, n-1\}$
não compartilham fatores primos com $n$.

\subsection{Por que os expoentes são determinados módulo $\varphi(n)$, e não módulo $n$?}

$\varphi(n)$ é a ordem do grupo multiplicativo módulo $n$
($\mathbb Z_n^*$).
Pelo teorema de Lagrange,
a ordem de qualquer elemento $m$ deste grupo divide a ordem do grupo;
portanto, $\varphi(n)$ é um múltiplo da ordem de todos elementos.
Assim, temos
\begin{equation*}
    m^{\varphi(n)} \equiv 1 \pmod n;
\end{equation*}
portanto,
\begin{equation*}
    m^{a\varphi(n) + b} \equiv m^b \pmod n.
\end{equation*}

Portanto, podemos calcular o expoente módulo $\varphi(n)$
antes de fazer a exponenciação.

\subsection{O que é PKCS \#1?}

É um conjunto de recomendações publicado pela RSA Laboratories
que contém normas de segurança relativas à implementação do algoritmo RSA.
Ele contém orientações
de como usar o RSA para cifrar dados em binário
(como fazer o \emph{padding}, por exemplo),
protocolos de assinatura digital,
e quais funções de hash devem ser usadas.

\subsection{Como o RSA pode ser usado para ciframento de dados? E como seria usado para assinatura?}

O processo de cifragem foi descrito na primeira seção deste texto.
O RSA é uma cifra assimétrica.
Tipicamente, o RSA não é usado diretamente para fazer a cifragem;
ele é usado para proteger uma chave simétrica
(do AES, por exemplo),
que será usado para cifrar os dados.

Para assinatura,
basta perceber que a escolha de qual chave é pública e qual é privada
é irrelevante para o algoritmo.
Portanto,
podemos ``cifrar'' um dado usando a chave privada
(tipicamente, apenas um resumo --- hash --- do dado em questão).
Quem tiver posse da chave pública,
ao ``decifrar'' o texto gerado com a chave privada,
obterá o texto original.
Como fabricar um texto que, ao ser ``decifrado'', gere o texto original,
é equivalente a quebrar o RSA,
temos uma boa confiança de que aquele texto assinado foi gerado pelo signatário.

\subsection{Quais ataques são conhecidos do RSA?}

De acordo com a Wikipédia~\cite{WikipediaRSA},
a melhor criptoanálise do algoritmo
(usando apenas informação pública)
é o algoritmo de fatoração GNFS (\emph{generalized field number sieve}).
Mesmo usando extenso poder computacional,
é difícil passar dos 500 bits
(o melhor até hoje é a fatoração de um número de 768 bits,
após dois anos de computação).

Do ponto de vista teórico,
existem alguns ataques que permitem obter o texto em claro sob algumas circunstâncias.
Por exemplo,
se tanto o texto $m$ quanto o expoente público $e$ forem pequenos,
isto é, $m^e < n$,
a exponenciação não extrapolou o valor de $n$,
portanto, usando algoritmos de extração de raízes de números inteiros,
podemos recuperar $m$ de forma rápida.

Outro ataque engenhoso envolve conseguir que o dono da chave privada
decifre um texto para nós.
O algoritmo do RSA satisfaz a propriedade de que
o produto das cifras é a cifra dos produtos.
Isto é,
\begin{equation*}
    m_1^e m_2^e \equiv (m_1 m_2)^e \pmod n.
\end{equation*}
Portanto,
dado um texto cifrado $m^e$,
podemos construir outro texto cifrado $r^e \bmod n$ usando algum número $r$
(lembrando que $e$ e $n$ são públicos)
e multiplicá-los, obtendo o valor insuspeito $c = m^e r^e = (mr)^e$,
Ao decifrá-lo, obtemos o valor $mr \bmod n$.
Se o valor $r$ for construído de forma a possuir inversa modular,
podemos calcular $m$ multiplicando $mr$ pela inversa modular de $r$.

\subsection{Por que é importante ter-se bons geradores de números aleatórios para gerar chaves RSA?}

Pelo mesmo motivo que bons geradores de números aleatórios
são necessários em qualquer aplicação criptográfica que use números aleatórios:
minimizar a possibilidade de se encontrar um padrão nos valores gerados
--- isto é, produzir o máximo de caos possível.

A Wikipédia menciona uma análise feita por Lenstra et al.:
se, ao gerar duas chaves $pq$ e $p' q'$,
por acaso $p = p'$,
então $\gcd(pq, p' q') = p$,
portanto ambas as chaves podem ser fatoradas rapidamente
caso o gerador de números aleatórios entre em ciclo,
ou as sementes iniciais são muito parecidas e o gerador
não foi projetado para lidar com esta situação.

\bibliographystyle{plain}
\bibliography{bibliografia}

\end{document}
