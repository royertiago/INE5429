% Slides que apareceram no vídeo da disciplina.
\documentclass{beamer}
\usetheme[compress]{Singapore}

\usepackage[utf8]{inputenc}
\usepackage[T1]{fontenc}
\usepackage[brazil]{babel}

\usepackage{tikz}

\begin{document}

\begin{frame}
    Exemplo: segredo compartilhado na União Soviética
\end{frame}

\begin{frame}
    \frametitle{Esquema de Shamir: esquema de limiar $(2, 3)$}
    \centering
    \begin{tikzpicture}
        \draw[->] (-1, 0) -- (5, 0);
        \draw[->] (0, -1) -- (0, 5);
        \foreach \x in {1, 2, 3, 4} {
            \draw (\x, 0.1) -- (\x, -0.1);
            \draw (0.1, \x) -- (-0.1, \x);
        }

        \onslide<2-> \draw[thick] (-1, 5) -- (5, -1);
        \onslide<1-> \fill[red] (0, 4) circle [radius=0.1cm];

        \onslide<3-> \fill[green] (1, 3) circle [radius=0.075cm];
        \onslide<3-> \fill[green] (2, 2) circle [radius=0.075cm];
        \onslide<4-> \fill[green] (3, 1) circle [radius=0.075cm];
    \end{tikzpicture}
\end{frame}

\begin{frame}
    \frametitle{Esquema de Shamir: esquema de limiar $(t, n)$}

    Escolha um polinômio $p(x) = a_0 + a_1 x^1 + \cdots + a_{t-1}x^{t-1}$

    Fatias: $(i, p(i)), i = 1, \ldots, n$

    Segredo: $p(0) = a_0$
\end{frame}

\begin{frame}
    \frametitle{Esquema de Cachin-Pinch}

    Seja $\mathcal P = \{p_1, \dots, p_n\}$ o conjunto de participantes,

    $p$ um primo grande e $f: \mathbb Z_p^* \to \mathbb Z_p^*$ uma função de via única.

    \onslide<2-> Escolha um segredo $S \in \mathbb Z_p^*$
    e combine, com cada participante $p_i$,
    uma fatia $s_i \in \mathbb Z_p^*$.

    \onslide<3-> Para cada conjunto $X \subseteq \mathcal P$
    de participantes autorizados,
    escolha um gerador $g_X$ para $\mathbb Z_p^*$
    e calcule
    \begin{align*}
        V_X &= g_X^{\prod_{P_i \in X} s_i} \\
        T_X &= S - f( V_X )
    \end{align*}
    e disponibilize o par $(g_X, T_X)$ no quadro de avisos.
\end{frame}

\begin{frame}
    \frametitle{Reconstrução do segredo}
    Suponha $X = \{p_1, p_2, \ldots, p_t\}$.
    \begin{enumerate}
        \item<2-> $p_1$ escolhe um número aleatório $\alpha \in \mathbb Z_p^*$
            e manda $g_X^{\alpha s_1}$ para $p_2$
        \item<3-> $p_2$ recebe $g_X^{\alpha s_1}$ de $p_1$,
            calcula $(g_X^{\alpha s_1})^{s_2} = g_X^{\alpha s_1 s_2}$
            e manda para $p_3$ \\
            ...
        \item<4-> $p_t$ recebe $g_X^{\alpha s_1 s_2 \dots s_{t-1}}$,
            calcula $g_X^{\alpha s_1 \dots s_t}$,
            e manda de volta para $p_1$
        \item<5-> $p_1$ calcula
            \begin{equation*}
                V_X = (g_X^{\alpha s_1 \dots s_t})^{\alpha^{-1}},
            \end{equation*}
            resgata $T_X$ do quadro de avisos e,
            em nome do grupo,
            reconstroi $S = T_X + f(V_X)$.
    \end{enumerate}
\end{frame}

\begin{frame}

    \texttt{github.com/royertiago/INE5429}

    \texttt{github.com/royertiago/INE5429/blob/master/final/e2.tex}

\end{frame}

\end{document}
