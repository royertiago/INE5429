\documentclass[10pt]{article}

\usepackage[utf8]{inputenc}
\usepackage[T1]{fontenc}
\usepackage[brazil]{babel}

\usepackage{amsmath}
\usepackage{amssymb}

\begin{document}

\title{
    INE5429 --- Segurança em Computação \\[1ex]
    Protocolo de segredo compartilhado \\
    que permite a alteração da estrutura de acesso \\[1ex]
    \makebox{Código produzido}
}
\author{Tiago Royer}
\date{23 de novembro de 2015}
\maketitle

\begin{abstract}
    O esquema de Pinch~\cite{Pinch1996}
    é um protocolo de segredo compartilhado
    que permite alteração da estrutura de acesso.
    No trabalho anterior,
    foi proposta a implementação do esquema de Pinch
    para uma estrutura de acesso de limiar.
    Como o trabalho anterior já descrevia a implementação por completo,
    este texto contém apenas o código-fonte escrito.

    Todo o código está disponível em \verb"https://github.com/royertiago/INE5429".
\end{abstract}

Entrada e saída e a interface com o usuário foram omitidas.

As classes \verb"dealer_information" e \verb"share"
dependem das classes \verb"noticeboard" e \verb"message",
portanto serão apresentadas por último.

\section{Quadro de avisos}

Código essencial de \verb"pinch/noticeboard.hpp".
Objetos da classe \verb"noticeboard" são os que constituem o arquivo
\verb"noticeboard.data".

\begin{verbatim}
template< typename T >
struct group_data {
    T group_generator;
    T group_value;
    std::vector<int> group;
};

template< typename T >
struct noticeboard {
    T generator, prime_modulo; // Parameters for the function f
    std::vector< group_data<T> > groups;

    // Retrieves the group_data corresponding to the given group.
    group_data<T> retrieve_data( std::vector<int> group ) const;
};

template< typename T >
group_data<T> noticeboard<T>::retrieve_data( std::vector<int> group ) const {
    for( int i = 0; i < groups.size(); i++ )
        if( groups[i].group == group )
            return groups[i];
    throw std::out_of_range( "Specified group not in the noticeboard." );
}
\end{verbatim}

\section{Mensagens trocadas}

Código de \verb"pinch/user_data.hpp".
A classe \verb"message" é a que vira o arquivo \verb"message.data",
e a classe \verb"private_nonce" vira \verb"r.data".

\begin{verbatim}
template< typename T >
struct message {
    std::vector<int> remaining_ids;
    T group_data;
    T partial_message;
    T prime_modulo;

    T generator; // Generator to the one-way function
};

template< typename T >
struct private_nonce {
    T nonce_inverse;

    /* Reconstruct the secret using a complete message.
     */
    T reconstruct( const message<T> & ) const;
};

template< typename T >
T private_nonce<T>::reconstruct( const message<T> & msg ) const {
    if( msg.remaining_ids.size() != 0 )
        throw std::invalid_argument( "The message must be final." );

    T V_X = math::pow_mod( msg.partial_message, nonce_inverse, msg.prime_modulo );
    T f_V_X = math::pow_mod( msg.generator, V_X, msg.prime_modulo );
    return (msg.group_data + f_V_X) % msg.prime_modulo;
}
\end{verbatim}

\section{Quota}

Código de \verb"pinch/shares.hpp".
A classe \verb"share" constitui cada um dos arquivos \verb"share"$i$\verb".data".

\begin{verbatim}
template< typename T >
struct share {
    int id;
    T share;

    /* Start the reconstruction of the shares,
     * using the given RNG to generate the secret nonce.
     *
     * The user list must not include this ID.
     */
    template< typename RNG >
    std::pair< message<T>, private_nonce<T> > start_reconstruction(
        const noticeboard<T> & board,
        std::vector< int > users,
        RNG & rng
    ) const;

    /* Join this share to the given message.
     */
    void join( message<T> & ) const;
};

template< typename T >
template< typename RNG >
std::pair< message<T>, private_nonce<T> > share<T>::start_reconstruction(
    const noticeboard<T> & board,
    std::vector< int > users,
    RNG & rng
) const {
    message<T> msg;
    private_nonce<T> nonce_holder;
    // First, generate the random nonce.
    T phi = board.prime_modulo - 1;
    int bits = mpz_sizeinbase( phi.get_mpz_t(), 2 );
    T nonce = rng::gmp_generate( rng, bits );
    while( math::gcd( nonce, phi ) != 1 )
        nonce = rng::gmp_generate( rng, bits );

    nonce_holder.nonce_inverse = math::modular_inverse( nonce, phi );

    // Next, retrieve the public information about the user list.
    msg.remaining_ids = users;
    users.push_back( id );
    std::sort( users.begin(), users.end() );
    group_data<T> data = board.retrieve_data( users );

    msg.group_data = data.group_value;
    msg.prime_modulo = board.prime_modulo;
    msg.generator = board.generator;

    // And now, give our contribution to the partial share.
    msg.partial_message = math::pow_mod(
            data.group_generator,
            T( nonce * this->share ),
            msg.prime_modulo
        );

    return std::make_pair( msg, nonce_holder );
}

template< typename T >
void share<T>::join( message<T> & msg ) const {
    for( int i = 0; i < msg.remaining_ids.size(); i++ )
        if( msg.remaining_ids[i] == id ) {
            msg.partial_message = math::pow_mod(
                    msg.partial_message,
                    this->share,
                    msg.prime_modulo
                );
            msg.remaining_ids.erase( msg.remaining_ids.begin() + i );
            return;
        }
    throw std::out_of_range( "The share of this id is not on the message's group." );
}
\end{verbatim}

\section{Informações do dealer}

Código de \verb"pinch/dealer_information.hpp".
Esta classe compõe o arquivo \verb"shares.data".

\begin{verbatim}
template< typename T >
struct dealer_information {
    std::vector< share<T> > valid_shares;
    int last_id = 0;
    T prime;
    T generator;

    /* Generates a new share for the list.
     */
    template< typename RNG >
    share<T> new_share( RNG & );

    /* Remove the share with the specified identifier.
     * To ensure that the specified user is not able to access the secret,
     * the noticeboard must be regenerated with another secret
     * after running this function.
     *
     * Returns true if something was removed, and false otherwise.
     */
    bool remove_share( int id );

    /* Generate a noticeboard for the given secret and the given threshold.
     * That is, every group with at least threshold members
     * will be able to reconstruct the given secret
     * using the information in the noticeboard.
     *
     * The RNG will be used to create the generators.
     */
    template< typename RNG >
    noticeboard<T> generate_noticeboard( T secret, int threshold, RNG & rng ) const;

};

template< typename T >
template< typename RNG >
share<T> dealer_information<T>::new_share( RNG & rng ) {
    // TODO: Make this GMP-independent
    int bits = mpz_sizeinbase( prime.get_mpz_t(), 2 );
    auto number = rng::gmp_generate( rng, bits );
    share<T> new_share{++last_id, T(number % prime)};
    valid_shares.push_back( new_share );
    return new_share;
}

template< typename T >
bool dealer_information<T>::remove_share( int id ) {
    for( int i = 0; i < valid_shares.size(); i++ )
        if( valid_shares[i].id == id ) {
            valid_shares.erase( valid_shares.begin() + i );
            return true;
        }
    return false;
}

template< typename T >
template< typename RNG >
noticeboard<T> dealer_information<T>::generate_noticeboard(
    T secret,
    int threshold,
    RNG & rng
) const {
    noticeboard<T> board;
    board.generator = generator;
    board.prime_modulo = prime;

    std::vector<int> indexes;
    for( int i = 0; i < valid_shares.size(); i++ )
        indexes.push_back( i );

    for( auto group_indexes: math::sorted_subsets( indexes, threshold ) ) {
        group_data<T> data;
        data.group_generator =
            math::random_primitive_root_modulo_p( prime, generator, rng );

        T power = 1; // Power that g_X must be raised to compute V_X.
        for( auto index : group_indexes ) {
            power *= valid_shares[index].share;
            data.group.push_back( valid_shares[index].id );
        }

        T V_X = math::pow_mod( data.group_generator, power, prime );
        T f_V_X = math::pow_mod( generator, V_X, prime );
        data.group_value = (secret - f_V_X + prime) % prime;

        board.groups.push_back( data );
    }

    return board;
}
\end{verbatim}

\bibliographystyle{plain}
\bibliography{bibliografia}

\end{document}
