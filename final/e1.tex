\documentclass[10pt,twocolumn]{article}
\usepackage[utf8]{inputenc}
\usepackage[T1]{fontenc}
\usepackage[brazil]{babel}

\begin{document}

\title{
    INE5429 --- Segurança em Computação \\[1ex]
    Protocolo de segredo compartilhado \\
    que permite a alteração da estrutura de acesso \\[1ex]
    \makebox{Descrição do tema e proposta de trabalho prático}
}
\author{Tiago Royer}
\date{2 de novembro de 2015}
\maketitle

\begin{abstract}
    Um protocolo de segredo compartilhado
    é um método de dividir um segredo $s$
    (uma chave criptográfica, por exemplo)
    entre vários participantes,
    de forma que seja necessário algum conjunto específico destas partes
    para que possamos reconstruir $s$.
    Por exemplo,
    num esquema de segredo compatrilhado com limiar $t$,
    são necessárias ao menos $t$ ``fatias'' do segredo $s$
    para que ele seja reconstruído;
    conhecimento de apenas $t-1$ fatias não é suficiente para reconstruir $s$.

    Neste trabalho,
    estudaremos um protocolo de segredo compartilhado
    que permite adicionar e remover participantes da estrutura de acesso,
    sem que haja a necessidade de redistribuir as partes de $s$.
\end{abstract}

\section{Introdução}
\section{Esquema de Shamir}
\section{Limitações inerentes ao esquema}
\section{Esquema de Pinch}
\section{Limitações do esquema de Pinch}
\bibliographystyle{plain}
\bibliography{bibliografia}

\end{document}
