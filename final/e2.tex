\documentclass[10pt,twocolumn]{article}

\usepackage[utf8]{inputenc}
\usepackage[T1]{fontenc}
\usepackage[brazil]{babel}

\usepackage{amsmath}
\usepackage{amssymb}

\begin{document}

\title{
    INE5429 --- Segurança em Computação \\[1ex]
    Protocolo de segredo compartilhado \\
    que permite a alteração da estrutura de acesso \\[1ex]
    \makebox{Descrição da implementação}
}
\author{Tiago Royer}
\date{17 de novembro de 2015}
\maketitle

\begin{abstract}
    O esquema de Pinch~\cite{Pinch1996}
    é um protocolo de segredo compartilhado
    que permite alteração da estrutura de acesso.
    No trabalho anterior,
    foi proposta a implementação do esquema de Pinch
    para uma estrutura de acesso de limiar.
    Este trabalho descreve os detalhes da implementação
    e mostra um exemplo de execução dos programas implementados.

Todo o código está disponível em \verb"https://github.com/royertiago/INE5429".
\end{abstract}

\section{Decisões de projeto}

Esta seção documenta as principais decisões de projeto
que afetaram a implementação do protocolo.


\subsection{Função de via única usada}

Seja $p$ o primo sobre o qual as operações do esquema de Pinch serão realizadas.
Um grupo $X$ de usuários autorizados pode construir o valor $V_X$ dado por
\begin{equation*}
    V_X = g_X^{\prod_{P_i \in X} s_i},
\end{equation*}
em que $s_i$ são as quotas de cada usuário
e $g_X$ é um gerador de $\mathbb Z_p^*$, escolhido aleatoriamente
e fixado no quadro de avisos.

O valor $T_X$,
que será publicado no quadro de avisos e permitirá a recuperação da chave,
é dado por
\begin{equation*}
    T_X = S - f(V_X).
\end{equation*}
$S$ é o segredo.

Todas essas operações são feitas no conjunto $\mathbb Z_p$,
portanto, a função de via única $f$ também precisa operar neste conjunto.
Como sabemos que resolver instâncias do logaritmo discreto é,
em geral, difícil,
optei por escolher $f$ como uma função de exponenciação modular em $\mathbb Z_p$.
É fixado um gerador $g$ para o conjunto $\mathbb Z_p^*$,
que será armazenado junto do quadro de avisos;
então, $f$ é definida por
\begin{equation*}
    f(n) = g^n \pmod p.
\end{equation*}

\subsection{Classes implementadas}

Seis classes foram implementadas para o trabalho.
O arquivo \verb"pinch/shares.hpp" contém a classe \verb"share",
o arquivo \verb"pinch/dealer_information.hpp" contém a classe \verb"dealer_information",
o arquivo \verb"pinch/user_data.hpp" contém a classe \verb"message"
e a clase \verb"private_nonce"
e o arquivo \verb"pinch/noticeboard.hpp" contém a classe \verb"group_data"
e \verb"noticeboard".

Apenas a classe \verb"group_data" não é armazenada separadamente num arquivo;
todas as demais podem ser serializadas individualmente.
Para facilitar a visualização do protocolo,
todos os arquivos contém os números serializados em formato texto,
ao invés de usar binário.

\begin{itemize}
    \item \verb"share":
        Armazena a quota de um usuário.
        Para facilitar a troca de mensagens,
        esta classe armazena não apenas a quota,
        mas também o ID deste usuário.

        Esta classe é responsável por atualizar a mensagem
        com a informação de sua quota.

    \item \verb"dealer_information":
        Contém toda a informação necessária para o dealer atualizar o quadro de avisos.
        É responsável por gerar o quadro de avisos
        e adicionar e remover quotas.

    \item \verb"message":
        Armazena as informações intermediárias necessárias
        para reconstruir o valor $V_X$.
        É gerada pela classe \verb"share".
        Para garantir desacoplamento do quadro de avisos,
        o primo $p$ (usado como módulo)
        e o gerador $g$ (da função $f$)
        também são armazenados na mensagem.
        Para guiar a reconstrução,
        esta classe também contém uma lista de todos os usuários
        que faltam contribuir com sua quota.

    \item \verb"private_nonce":
        Contém a inversa modular
        do número aleatório gerado pelo usuário que iniciou a reconstrução.
        (Este número aleatório é usado para proteger a quota
        do primeiro usuário.)
        Esta é a última peça do quebra-cabeça após todas as quotas terem sido unidas;
        portanto, esta é a classe responsável por finalizar a reconstrução do segredo.
        
    \item \verb"group_data":
        Esta é uma classe interna do quadro de avisos
        que representa o par $(g_X, T_X)$ para cada grupo $X$.
        Também contém a lista dos IDs dos membros do grupo $X$.

    \item \verb"noticeboard":
        Representa o quadro de avisos.
        Armazena o gerador $g$ (para a função $f$),
        o primo $p$,
        e uma lista de \verb"group_data",
        contendo todos os conjuntos do quadro de avisos.
\end{itemize}

\section{Exemplo de execução}

Nesta seção,
será repetido o exemplo de execução do trabalho anterior,
mas agora discutindo as estruturas geradas.

\subsection{Construção dos dados do dealer}

O dealer possuirá um banco de dados com as informações dos usuários
--- a saber, o ID e a quota de cada um.

\begin{verbatim}
$ pinch_dealer shares.data --prime 691 \
  --generator 3
$ pinch_dealer shares.data --add share1.data
$ pinch_dealer shares.data --add share2.data
$ pinch_dealer shares.data --add share3.data
$ pinch_dealer shares.data --add share4.data
$ pinch_dealer shares.data --add share5.data
\end{verbatim}
Estes comandos criam o arquivo \verb"shares.data"
(que contém todos os dados que o dealer precisa para gerenciar o segredo),
\verb"share1.data" (a quota do usuário com ID $1$),
\verb"share2.data" (a quota do usuário com ID $2$),
$\ldots$,
e \verb"share5.data" (a quota do usuário com ID $5$).

Neste exemplo de execução, o arquivo \verb"shares.data" contém
\begin{verbatim}
5 691 3
1 546
2 237
3 91
4 651
5 651
\end{verbatim}
Os três primeiros valores são o último ID,
o primo $p$ e o gerador $g$.
Então, cada linha contém uma quota;
por exemplo, a quota do usuário com ID $1$ é $546$.
Cada um dos arquivos \verb$share$$i$\verb$.data$
contém o par $(ID, quota)$ para cada usuário;
por exemplo,
\verb"share1.data" contém
\begin{verbatim}
1 546
\end{verbatim}

O arquivo \verb"shares.data" deve ser mantido privado pelo dealer;
os arquivos \verb"share"$i$\verb".data" devem ser enviados para os respectivos usuários
através de um canal seguro de comunicação.
(Uma implementação mais rebuscada poderia usar o protocolo Diffie-Hellman
para estabelecer os valores das quotas de cada usuário;
escolhi fazer assim para simplificar a execução.)

\subsection{Geração do quadro de avisos}

\begin{verbatim}
$ pinch_dealer shares.data --threshold 3 \
  --secret 189 --noticeboard noticeboard.data
\end{verbatim}
Este comando armazena no arquivo \verb"noticeboard.data" o quadro de avisos.
Este quadro ocultará o segredo $189$,
e exigirá três usuários para reconstruir o segredo.

Neste exemplo, o arquivo \verb"noticeboard.data" contém
\begin{verbatim}
3 691
126 95 3 1 2 3
251 108 3 1 2 4
296 616 3 1 2 5
461 147 3 1 3 4
461 147 3 1 3 5
472 222 3 1 4 5
186 547 3 2 3 4
296 216 3 2 3 5
92 2 3 2 4 5
17 245 3 3 4 5
\end{verbatim}
Na primeira linha, temos o gerador $g$ e o primo $p$.
Cada uma das linhas subsequentes representa um conjunto $X$;
por exemplo, a segunda linha,
\begin{verbatim}
126 95 3 1 2 3
\end{verbatim}
contém, em sequência, os valore $T_X$, $g_X$ e $|X|$,
seguidos dos IDs dos usuários do conjunto $X$.
Estritamente falando,
não é necessário armazenar $|X$ nem os IDs dos usuários do conjunto $X$,
mas isso facilita o controle sobre a ordem do protocolo.

O quadro de avisos pode ser publicado livremente em algum lugar com alta disponibilidade
ou mandado para cada usuário (usando um canal inseguro mesmo).

\subsection{Reconstrução do segredo}

\subsubsection{Início da reconstrução}

\begin{verbatim}
$ pinch_user share1.data --reconstruct \
  noticeboard.data message.data r.data \
  2 5
Generated message.data.
Send this file to user 2 to continue.
\end{verbatim}
Este comando inicia o processo de reconstrução.
\verb"noticeboard.data" é o quadro de avisos,
que é apenas lido neste comando.
\verb"message.data" armazena quase todas as informações necessárias para a reconstrução
--- a única parte que falta é o \emph{nonce} gerado pelo primeiro usuário,
que é armazenado no arquivo \verb"r.data".

Nesse exemplo,
\verb"message.data" contém
\begin{verbatim}
616 641 691 3
2 5
\end{verbatim}
e \verb"r.data" contém
\begin{verbatim}
133
\end{verbatim}
Os dois primeiros valores de \verb"message.data" são
$T_X$ e $g_X^{r s_1}$.
Estes valores são efetivamente parte exclusiva da mensagem.
Os dois seguintes são $p$ e $g$;
embora estes valores possam ser resgatados do quadro de avisos,
trazê-los junto na mensagem faz com que todo o processo de reconstrução do segredo
seja, agora, independente do arquivo \verb"noticeboard.data".

A última linha contém os IDs dos usuários que ainda não contribuiram com sua quota.

\verb"r.data" contém a inversa modular de $r$.
Este arquivo deve ser mantido privado pelo usuário $1$;
o arquivo \verb"message.data" pode ser enviado usando um canal inseguro
para os demais usuários.

\subsubsection{Juntando os demais pedaços}

\begin{verbatim}
$ pinch_user share2.data --join message.data
Joined your share to message.data.
Send this file to user 5 to continue.
\end{verbatim}
Assim, o usuário $2$ contribuiu com sua quota para a reconstrução.
O conteúdo de \verb"message.data" é
\begin{verbatim}
616 165 691 3
5
\end{verbatim}

\begin{verbatim}
$ pinch_user share5.data --join message.data
Joined your share to message.data.
Return this file back to the starter to finish.
\end{verbatim}

O conteúdo de \verb"message.data" é
\begin{verbatim}
616 389 691 3

\end{verbatim}
Observe que a última linha está vazia.

\subsubsection{Reconstrução final}

\begin{verbatim}
$ pinch_user --finish-reconstruction \
  message.data r.data
Secret: 189
\end{verbatim}

O usuário $1$, que detinha \verb"r.data", pôde agora reconstruir o segredo.

\subsubsection{Alteração na estrutura de acesso}

\begin{verbatim}
$ pinch_dealer shares.data --add share6.data
$ pinch_dealer shares.data --remove 2
$ pinch_dealer shares.data --threshold 3 \
  --secret 196 --noticeboard noticeboard.data
\end{verbatim}

Estes comandos adicionam mais um usuário (o usuário $6$)
e removem o usuário $2$ do arquivo \verb"shares.data".
Assim,
quando o dealer reconstruir o quadro de avisos,
nenhum grupo $X$ conterá o usuário $2$,
portanto ele não poderá participar de uma reconstrução do segredo.

\bibliographystyle{plain}
\bibliography{bibliografia}

\end{document}
