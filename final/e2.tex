\documentclass[10pt,twocolumn]{article}

\usepackage[utf8]{inputenc}
\usepackage[T1]{fontenc}
\usepackage[brazil]{babel}

\usepackage{amsmath}
\usepackage{amssymb}

\begin{document}

\title{
    INE5429 --- Segurança em Computação \\[1ex]
    Protocolo de segredo compartilhado \\
    que permite a alteração da estrutura de acesso \\[1ex]
    \makebox{Descrição da implementação}
}
\author{Tiago Royer}
\date{17 de novembro de 2015}
\maketitle

\begin{abstract}
    O esquema de Pinch~\cite{Pinch1996}
    é um protocolo de segredo compartilhado
    que permite alteração da estrutura de acesso.
    No trabalho anterior,
    foi proposta a implementação do esquema de Pinch
    para uma estrutura de acesso de limiar.
    Este trabalho descreve os detalhes da implementação
    e mostra um exemplo de execução dos programas implementados.

Todo o código está disponível em \verb"https://github.com/royertiago/INE5429".
\end{abstract}

\section{Decisões de projeto}

Esta seção documenta as principais decisões de projeto
que afetaram a implementação do protocolo.


\subsection{Função de via única usada}

Seja $p$ o primo sobre o qual as operações do esquema de Pinch serão realizadas.
Um grupo $X$ de usuários autorizados pode construir o valor $V_X$ dado por
\begin{equation*}
    V_X = g_X^{\prod_{P_i \in X} s_i},
\end{equation*}
em que $s_i$ são as quotas de cada usuário
e $g_X$ é um gerador de $\mathbb Z_p^*$, escolhido aleatoriamente
e fixado no quadro de avisos.

O valor $T_X$,
que será publicado no quadro de avisos e permitirá a recuperação da chave,
é dado por
\begin{equation*}
    T_X = S - f(V_X).
\end{equation*}
$S$ é o segredo.

Todas essas operações são feitas no conjunto $\mathbb Z_p$,
portanto, a função de via única $f$ também precisa operar neste conjunto.
Como sabemos que resolver instâncias do logaritmo discreto é,
em geral, difícil,
optei por escolher $f$ como uma função de exponenciação modular em $\mathbb Z_p$.
É fixado um gerador $g$ para o conjunto $\mathbb Z_p^*$,
que será armazenado junto do quadro de avisos;
então, $f$ é definida por
\begin{equation*}
    f(n) = g^n \pmod p.
\end{equation*}

\subsection{Classes implementadas}

Seis classes foram implementadas para o trabalho.
O arquivo \verb"pinch/shares.hpp" contém a classe \verb"share",
o arquivo \verb"pinch/dealer_information.hpp" contém a classe \verb"dealer_information",
o arquivo \verb"pinch/user_data.hpp" contém a classe \verb"message"
e a clase \verb"private_nonce"
e o arquivo \verb"pinch/noticeboard.hpp" contém a classe \verb"group_data"
e \verb"noticeboard".

Apenas a classe \verb"group_data" não é armazenada separadamente num arquivo;
todas as demais podem ser serializadas individualmente.
Para facilitar a visualização do protocolo,
todos os arquivos contém os números serializados em formato texto,
ao invés de usar binário.

\begin{itemize}
    \item \verb"share":
        Armazena a quota de um usuário.
        Para facilitar a troca de mensagens,
        esta classe armazena não apenas a quota,
        mas também o ID deste usuário.

        Esta classe é responsável por atualizar a mensagem
        com a informação de sua quota.

    \item \verb"dealer_information":
        Contém toda a informação necessária para o dealer atualizar o quadro de avisos.
        É responsável por gerar o quadro de avisos
        e adicionar e remover quotas.

    \item \verb"message":
        Armazena as informações intermediárias necessárias
        para reconstruir o valor $V_X$.
        É gerada pela classe \verb"share".
        Para garantir desacoplamento do quadro de avisos,
        o primo $p$ (usado como módulo)
        e o gerador $g$ (da função $f$)
        também são armazenados na mensagem.
        Para guiar a reconstrução,
        esta classe também contém uma lista de todos os usuários
        que faltam contribuir com sua quota.

    \item \verb"private_nonce":
        Contém a inversa modular
        do número aleatório gerado pelo usuário que iniciou a reconstrução.
        (Este número aleatório é usado para proteger a quota
        do primeiro usuário.)
        Esta é a última peça do quebra-cabeça após todas as quotas terem sido unidas;
        portanto, esta é a classe responsável por finalizar a reconstrução do segredo.
        
    \item \verb"group_data":
        Esta é uma classe interna do quadro de avisos
        que representa o par $(g_X, T_X)$ para cada grupo $X$.
        Também contém a lista dos IDs dos membros do grupo $X$.

    \item \verb"noticeboard":
        Representa o quadro de avisos.
        Armazena o gerador $g$ (para a função $f$),
        o primo $p$,
        e uma lista de \verb"group_data",
        contendo todos os conjuntos do quadro de avisos.
\end{itemize}

\bibliographystyle{plain}
\bibliography{bibliografia}

\end{document}
